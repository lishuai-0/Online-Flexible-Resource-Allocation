\chapter{Conclusion and future work}
\label{ch:conclusion-and-future-work}
The aim of this project was to expand previous research to fix flaws in the Flexible resource allocation optimisation
problems for Mobile Edge Computing. This was achieved by introducing the notion of time into the optimisation
model. As a result, a new optimisation problem was presented in Section~\ref{sec:resource-allocation-optimisation-problem}
along with an auction mechanism to deal with self-interested users (Section~\ref{sec:auctioning-of-tasks}). \\
Due to the complexity of the problem, Reinforcement learning was applied to learn how to bid on auctioned tasks and
allocation the server's resources. By implementing an MEC environment and numerous Reinforcement Learning algorithms,
agents were able to be train and learn the optimal strategy for both task pricing and resource allocation. \\
The evaluation and testing of the implementation, in Chapter~\ref{ch:testing-and-evaluation}, found that agents could
efficiently learn the policy achieving significantly more tasks completed than the Fixed Resource Allocation mechanisms.
However around $\sim$ 5\% of all tasks were not completed within their time frame. As a result, this project has been
viewed as a success however more research and analysis of agents are required before such systems
can be implemented into real-life environments.

For future work into this project, this author believes that several additions to the proposed agents could greatly
improve their performance like n-step rewards~\citep{multi-step-dqn} and distributional ddpg
agents~\citep{distributional_dqn, d4pg} that would improve agents due to the environment stochastic nature for agents. An
additional heuristic for the policy gradient, would be to use a centralised critic~\citep{maddpg} that has been proposed
in mix competitive-cooperative environment to help multiple agents working together.

The word count of the Project can be found in \hyperref[app:project-management]{Appendix E}.
